\documentclass[11pt, a4paper, oneside]{article}
\usepackage{worksheet}
\usepackage{lipsum}

\begin{document}
	\begin{table}
	\hrule
	\vspace{.1cm}
	\setlength{\tabcolsep}{0cm}
	\renewcommand{\arraystretch}{1.5}
	\begin{tabular}{p{4cm} >{\Large\bfseries}p{8cm} p{4cm}}
		\multirow{2}{4cm}{L. Bung} & \multirow{2}{8cm}{\centering The worksheet package} & \multirow{2}{4cm}{\today} \\\\
	\end{tabular}
	\hrule
	\end{table}
	
	This document aims to provide an overview for the usage of the \texttt{worksheet} package.
	The goal of the package is to simplify the creation of optically pleasing worksheets and class tests used in school environments.
	
	Please note that the code is still in development and subject to -- possibly breaking -- changes.
	
	\section*{Document header}
	
	The header of the document uses the default \LaTeX\ command \texttt{\textbackslash maketitle}.
	However, the commands used to add information to the header vary from the defaults:
	\begin{itemize}[label=]
		\item \texttt{\textbackslash author}: Used to set the name of the teacher, such as ``John Doe''
		\item \texttt{\textbackslash date}: Mainly used for class tests, used to add a date to the document
		\item \texttt{\textbackslash title}: The topic of the worksheet -- generally, the title of the document
		\item \texttt{\textbackslash subject}: Used to set the subject of the class, for example ``Mathematics''
		\item \texttt{\textbackslash class}: The name of the class, such as 9a) or BG11
	\end{itemize}
	All of the above arguments can be left empty.
	
	The document header can be generated using \texttt{\textbackslash maketitle}.
	When creating a class test, the additional command \texttt{\textbackslash testheader[points]\{remarks\}} can be used to append a grading template to the header.
	The optional argument \texttt{points} can be used to specify the maximum amount of points that can be attained.
	In \texttt{remarks}, additional information about the test can be specified, such as available time or permitted aids.
	
	\section*{Tasks} 
	
	A key part of worksheets or class tests are the tasks.
	This package provides several commands for that:
	\begin{itemize}[label=]
		\item \texttt{\textbackslash singletask}: A task to be completed alone
		\item \texttt{\textbackslash partnertask}: A task to be done in pairs (not in groups!)
		\item \texttt{\textbackslash grouptask}: A task to be done in groups of multiple students
		\item \texttt{\textbackslash bonustask}: An additional task that can be done by fast or strong students, but doesn't have to
		\item \texttt{\textbackslash testtask}: For tasks in class tests
	\end{itemize}
	The only difference between the different task types is the icon generated along them.
	
	All task commands follow the same pattern: \texttt{\textbackslash tasktype[points]\{title\}}.
	The optional value provided in \texttt{points} can be used to add points to class tests, but also to add a time (or any text, for that matter) to a task.
	
	Each task command also comes with a starred version (e.g. \texttt{\textbackslash singletask*}), with the difference being that starred tasks won't be enumerated.
	Note that the counter also won't be incremented in that case.
	
	\section*{Writing areas}
	
	In many cases, tasks are supposed to be completed right on the worksheet or class test  paper itself.
	For that reason, space has to be provided for students to write or draw on.
	Currently, the following frequent types of writing areas are provided:
	\begin{itemize}[label=]
		\item \texttt{\textbackslash lines}: Basic lines spaced 1cm apart to give textual answers
		\item \texttt{\textbackslash checkered}: A checkered area for graphing, technical drawings or the like
		\item \texttt{\textbackslash boxarea}: A free-form box, usable for drawings, diagrams or anything else
	\end{itemize}
	All of the commands follow the pattern \texttt{\textbackslash areatype[length]}.
	The optional parameter \texttt{length} can be used to specify the vertical length of the area.
	Lines will get clipped in 1cm steps and checkered areas to 5mm steps, while box areas can have any length.
	In case no length is provided, the default length for writing areas is 4cm.
	All areas span have the width of \texttt{\textbackslash textwidth}.
	
	\section*{Hints and warnings}
	
	When creating information material, it is often useful to highlight certain things.
	There are currently the following options for that:
	\begin{itemize}[label=]
		\item \texttt{\textbackslash hint}: Hints for solving tasks or helpful information. Denoted by a lightbulb icon
		\item \texttt{\textbackslash warning}: Warnings and important alerts. Denoted by an exclamation mark
	\end{itemize}
	Again, the commands follow the same pattern: \texttt{\textbackslash command\{title\}\{body\}}.
	\texttt{title} will be printed in the first line next to the icon in bold text.
	The content of \texttt{body} is the text displayed inside of the box.
	
	\section*{Environments specific to computer science}
	
	There is a wide spectrum of \LaTeX\ packages for the various use cases of computer science.
	Although one of the paradigms of this package is to remain as minimal as possible, certain packages are included in order to add sensible defaults.
	The following list is supposed to show an overview of what is possible:
	\begin{enumerate}
		\item \textbf{Code listings}.
		The widespread package \texttt{listings}\footnote{\url{https://ctan.org/pkg/listings}} provides many useful options for code listings.
		Certain options, such as colors, line numbers and the like are set to default values by this package in order to blend in nicely with the rest of the document.
		\item \textbf{UML diagrams}.
		UML diagrams can be drawn with TikZ using the \texttt{pgf-umlcd}\footnote{\url{https://ctan.org/pkg/pgf-umlcd}} package.
	\end{enumerate}

	\section*{Environments specific to mathematics}
	
	Of course, \LaTeX\ already has great support for mathematical expressions out of the box.
	This functionality can be greatly expanded by the use of packages.
	Again, the following list only shows excerpts of what is possible:
	\begin{enumerate}
		\item \textbf{Vector-based graphics}.
		Instead of using image files, it is often preferable to use well-scaling vector-based graphics.
		In \LaTeX, this can be achieved using TikZ \& PGF\footnote{\url{https://www.ctan.org/pkg/tikz}}.
		The base functionality can also be expanded, for example for geometry purposes using \texttt{tkz-euclide}\footnote{\url{https://www.ctan.org/pkg/tkz-euclide}}.
		\item \textbf{Plotting}.
		To create TikZ-based plots, the package \texttt{pgfplots}\footnote{\url{https://www.ctan.org/pkg/pgfplots}} can be used.
		The \texttt{worksheet} package already creates some sensible defaults for plots, but it is often necessary to adjust them.
		Also, the additional environment \texttt{checkeredfigure} can be used to create TikZ graphics in front of a checkered background.
	\end{enumerate}
	
	\section*{Language support}
	
	Currently supported languages are german and english.
	The package checks for the loaded \texttt{babel} class and translates accordingly in the necessary regions.
	If the language is not recognized, it defaults to english.
	
	\pagebreak
	
	\author{John Doe}
	\title{Class Test \#1}
	\subject{Mathematics}
	\class{9a)}
	\date{31.05.2025}
	\maketitle
	\testheader[60]{Additional information for the class test go here.}

	\singletask{Lorem ipsum}
	
	\lipsum[1]
	
	\checkered
	
	\partnertask{Dolor sit amet}
	
	\lipsum[2]
	\begin{plot}
		\addplot[domain=0:4.5, samples=100, color=red]{x^2 - 4*x + 5} node[yshift=-2.5cm, xshift=-1cm]{$f(x)$};
		\addplot[color=blue, ]{1/2*x+1} node[below] (g) {$g(x)$};
	\end{plot}
	$$\int_{1}^{3} f(x) dx = \dots?$$
	\boxarea[3cm]
	
	\grouptask{consecetuer adipiscing elit}
	
	\lipsum[4]
	
	\lines[3cm]
	\lipsum[5]
	
	\warning{Warning: Ut purus elit, vestibulum ut}{\lipsum[1]}
	
	\singletask*{No task enumeration}
	
	\lipsum[4]
	
	\bonustask{An additional task}
	
	\hint{Hint}{\lipsum[1]}
	
	\testtask[(1+2+3 Punkte)]{A class test task with additional points awarded}
	
	\begin{lstlisting}[language=python]
sum = 0
for i in range(0,10):
    sum += i #this is a comment
print('The sum is ' + sum)
	\end{lstlisting}

	\lipsum[7]

	\begin{figure}[H]
		\centering
		\begin{tikzpicture}
			\begin{class}[text width=5cm]{SuperClass}{3,3}
				\attribute{+attribute1: String}
				\attribute{-attribute2: int}
				\operation{+operation(param1: int)}
			\end{class}
			\begin{class}[text width=5cm]{SubClass1}{0,0}
				\inherit{SuperClass}
				\attribute{+attribute3: bool}
				\operation{+operation2()}
			\end{class}
			\begin{class}[text width=5cm]{SubClass2}{6,0}
				\inherit{SuperClass}
				\attribute{+attribute4: float}
				\operation{+operation3(param1: String)}
			\end{class}
		\end{tikzpicture}
	\end{figure}
\end{document}
